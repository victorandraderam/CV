\documentclass[spanish,]{article}
\usepackage[utf8]{inputenc}  
\usepackage{fontspec}
\usepackage{fontawesome}
\setromanfont[
BoldFont=Calluna-Bold.otf,
ItalicFont=Calluna-It.otf,
BoldItalicFont=Calluna-BoldIt.otf,
]{Calluna-Regular.otf}
\setsansfont[
BoldFont=CallunaSans-Bold.otf,
ItalicFont=CallunaSans-Italic.otf,
BoldItalicFont=CallunaSans-BoldItalic.otf
]{CallunaSans-Regular.otf}
 \renewcommand{\familydefault}{\sfdefault}
\usepackage{amssymb,amsmath}
\usepackage{ifxetex,ifluatex}
\usepackage{fixltx2e} % provides \textsubscript
\ifnum 0\ifxetex 1\fi\ifluatex 1\fi=0 % if pdftex
  \usepackage[T1]{fontenc}
  \usepackage[utf8]{inputenc}
\else % if luatex or xelatex
  \ifxetex
    \usepackage{mathspec}
  \else
    \usepackage{fontspec}
  \fi
  \defaultfontfeatures{Ligatures=TeX,Scale=MatchLowercase}
\fi
% use upquote if available, for straight quotes in verbatim environments
\IfFileExists{upquote.sty}{\usepackage{upquote}}{}
% use microtype if available
\IfFileExists{microtype.sty}{%
\usepackage{microtype}
\UseMicrotypeSet[protrusion]{basicmath} % disable protrusion for tt fonts
}{}
\usepackage{hyperref}
\hypersetup{unicode=true,
            pdftitle={Participación Programa Summer Institute in Law and Economics},
            pdfborder={0 0 0},
            breaklinks=true}
\urlstyle{same}  % don't use monospace font for urls
\ifnum 0\ifxetex 1\fi\ifluatex 1\fi=0 % if pdftex
  \usepackage[shorthands=off,main=spanish]{babel}
\else
  \usepackage{polyglossia}
  \setmainlanguage[]{spanish}
\fi
\usepackage{longtable,booktabs}
\IfFileExists{parskip.sty}{%
\usepackage{parskip}
}{% else
\setlength{\parindent}{0pt}
\setlength{\parskip}{6pt plus 2pt minus 1pt}
}
\setlength{\emergencystretch}{3em}  % prevent overfull lines
\providecommand{\tightlist}{%
  \setlength{\itemsep}{0pt}\setlength{\parskip}{0pt}}
\setcounter{secnumdepth}{0}
% Redefines (sub)paragraphs to behave more like sections
\ifx\paragraph\undefined\else
\let\oldparagraph\paragraph
\renewcommand{\paragraph}[1]{\oldparagraph{#1}\mbox{}}
\fi
\ifx\subparagraph\undefined\else
\let\oldsubparagraph\subparagraph
\renewcommand{\subparagraph}[1]{\oldsubparagraph{#1}\mbox{}}
\fi

\title{Participación Programa \\ \textit{Summer Institute in Law and Economics}}
\providecommand{\subtitle}[1]{}
\subtitle{Comentarios Anteproyecto Reforma Ley 19.628}
\date{}
\author{\textsc{Víctor Andrade R.}}

\begin{document}
\maketitle

\subsection{1. Datos del Programa}\label{datos-del-programa}

\begin{longtable}[c]{@{}ll@{}}
\toprule
\begin{minipage}[t]{0.18\columnwidth}\raggedright\strut
Nombre del Programa:
\strut\end{minipage} &
\begin{minipage}[t]{0.76\columnwidth}\raggedright\strut
2016 Summer Institute in Law and Economics
\strut\end{minipage}\tabularnewline
\begin{minipage}[t]{0.18\columnwidth}\raggedright\strut
Universidad:
\strut\end{minipage} &
\begin{minipage}[t]{0.76\columnwidth}\raggedright\strut
University of Chicago - Coase Sandor Institute in Law and Economics
\strut\end{minipage}\tabularnewline
\begin{minipage}[t]{0.18\columnwidth}\raggedright\strut
Dirección:
\strut\end{minipage} &
\begin{minipage}[t]{0.76\columnwidth}\raggedright\strut
Omri Ben-Shahar (Phd \& SJD Harvard University) Faculty Director
Coase-Sandor Institute for Law and Economics
\strut\end{minipage}\tabularnewline
\begin{minipage}[t]{0.18\columnwidth}\raggedright\strut
Duración:
\strut\end{minipage} &
\begin{minipage}[t]{0.76\columnwidth}\raggedright\strut
11 al 22 de Julio de 2016
\strut\end{minipage}\tabularnewline
\begin{minipage}[t]{0.18\columnwidth}\raggedright\strut
Ciudad:
\strut\end{minipage} &
\begin{minipage}[t]{0.76\columnwidth}\raggedright\strut
Chicago, Illinois, USA
\strut\end{minipage}\tabularnewline
\begin{minipage}[t]{0.18\columnwidth}\raggedright\strut
Valor del Programa:
\strut\end{minipage} &
\begin{minipage}[t]{0.76\columnwidth}\raggedright\strut
USD 3000 (incluyendo depósito de seguridad de USD 500). Pasajes y gastos
por cuenta del participante
\strut\end{minipage}\tabularnewline
\begin{minipage}[t]{0.18\columnwidth}\raggedright\strut
Sistema de clases:
\strut\end{minipage} &
\begin{minipage}[t]{0.76\columnwidth}\raggedright\strut
4 unidades. Cada una con 5 cátedras de 2 horas de duración +
Presentación de Working Paper
\strut\end{minipage}\tabularnewline
\begin{minipage}[t]{0.18\columnwidth}\raggedright\strut
Tópicos:
\strut\end{minipage} &
\begin{minipage}[t]{0.76\columnwidth}\raggedright\strut
(a) \Behavioral Law and Economics (Tools and Methods), (b)
Economic Analysis of Corporate Law, (c) Empirical Methods
in Law and Economics Research, (d) Law and Economics of
Contracts
\strut\end{minipage}\tabularnewline
\begin{minipage}[t]{0.18\columnwidth}\raggedright\strut
\strut\end{minipage} &
\begin{minipage}[t]{0.76\columnwidth}\raggedright\strut
\strut\end{minipage}\tabularnewline
\bottomrule
\end{longtable}

\subsection{2. Expectativas y Objetivos
Esperados}\label{expectativas-y-objetivos-esperados}

Una de mis principales aspiraciones con el programa es perfeccionar mi
conocimiento sobre métodos de análisis normativo propios de la corriente
teórica denominada ``Análisis Económico del Derecho'' (\emph{Law and
Economics}), particularmente en vista de los recientes hallazgos en el
campo de la economía conductual (\emph{behavioral economics}) que
explican el comportamiento de los individuos ante decisiones económicas
complejas habida cuenta de los sesgos cognitivos a los que se encuentran
expuestos. Lo anterior, con el fin de aplicar el mismo a tópicos de
regulación financiera, especialmente la modernización de los medios y
sistemas de pagos.

Tal propósito obedece al hecho que, en mi opinión, uno de los aspectos
que definen la coyuntura actual en el campo de la regulación de medios
de pagos es la tensión inherente entre innovación financiera,
cuestiones de competencia, estabilidad y seguridad del sistema de pagos,
y la protección del consumidor financiero. En dicho contexto, resulta
necesario, junto con una profundización en el conocimiento sustantivo
sobre el funcionamiento de sistemas de pago, contar con herramientas de
evaluación y análisis legal que permitan enriquecer las labores de
proposición normativa como asimismo la coordinación entre agencias
gubernamentales.

Por otra parte, como parte del desarrollo del programa, espero presentar
en los coloquios de discusión un documento de trabajo (\emph{working
paper}) relativo a los alcances de la normativa chilena sobre interés
máximo convencional aplicada a operaciones con
tarjetas de crédito; trabajo que junto con ser discutido con académicos
y otros partícipes del curso, será eventualmente parte de un repositorio
de trabajos en línea del \emph{Coase Sandor Institute for Law and
Economics}.

\subsection{3. Adjuntos}\label{adjuntos}

\begin{enumerate}
\def\labelenumi{\arabic{enumi}.}
\tightlist
\item
  \emph{Brouchure} del programa.
\item
  Reseña de los relatores.
\item
  Carta de aceptación.
\end{enumerate}

\end{document}
