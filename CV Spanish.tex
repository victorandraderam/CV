
\documentclass[11pt,a4paper,sans]{moderncv}        % possible options include font size ('10pt', '11pt' and '12pt'), paper size ('a4paper', 'letterpaper', 'a5paper', 'legalpaper', 'executivepaper' and 'landscape') and font family ('sans' and 'roman')
% moderncv themes
\moderncvstyle{classic}                             % style options are 'casual' (default), 'classic', 'banking', 'oldstyle' and 'fancy'
\moderncvcolor{blue}                               % color options 'black', 'blue' (default), 'burgundy', 'green', 'grey', 'orange', 'purple' and 'red'
%\renewcommand{\familydefault}{\sfdefault}         % to set the default font; use '\sfdefault' for the default sans serif font, '\rmdefault' for the default roman one, or any tex font name
%\nopagenumbers{}                                  % uncomment to suppress automatic page numbering for CVs longer than one page
% character encoding
\usepackage[utf8]{inputenc}  
\usepackage{fontspec}
\usepackage{fontawesome}
\setromanfont[
BoldFont=Calluna-Bold.otf,
ItalicFont=Calluna-It.otf,
BoldItalicFont=Calluna-BoldIt.otf,
]{Calluna-Regular.otf}
\setsansfont[
BoldFont=CallunaSans-Bold.otf,
ItalicFont=CallunaSans-Italic.otf,
BoldItalicFont=CallunaSans-BoldItalic.otf
]{CallunaSans-Regular.otf}
\usepackage{natbib}
\usepackage[scale=0.75]{geometry}
\newcommand{\noun}[1]{\textsc{#1}}
% personal data
\name{\textsc{Víctor}}{\textsc{Andrade}}
\title{Curriculum Vitae}                               % optional, remove / comment the line if not wanted
\address{}{Santiago}{Chile}% optional, remove / comment the line if not wanted; the "postcode city" and "country" arguments can be omitted or provided empty
\phone[mobile]{+56~982400995}                   % optional, remove / comment the line if not wanted; the optional "type" of the phone can be "mobile" (default), "fixed" or "fax"
%\phone[fixed]{+56~223245766}
\email{contacto@victorandrade.cl}                               % optional, remove / comment the line if not wanted
% \homepage{www.victorandrade.cl}                         % 

\makeatletter\renewcommand*{\bibliographyitemlabel}{\@biblabel{\arabic{enumiv}}}\makeatother
\usepackage[spanish]{babel}
\begin{document}
%\begin{CJK*}{UTF8}{gbsn}                          % to typeset your resume in Chinese using CJK
%-----       resume       ---------------------------------------------------------
\makecvtitle

\section{\noun{Áreas de Interés}}

Regulación financiera, protección de datos personales, contratos de tecnología, riesgos cibernéticos, medios de pago e innovación financiera (Fintech), interoperabilidad de datos, Derecho del Consumo, comercio electrónico y plataformas, técnicas de medición de impacto regulatorio (RIA), y economía del comportamiento aplicada al análisis económico del derecho (\emph{behavioral law and economics}).

\section{\noun{Información Académica}}


\cventry{2015}{Corte Suprema}{}{}{}{Título de abogado}
\cventry{2015}{Universidad de Chile}{}{}{}{Licenciatura en Ciencias Jurídicas y Sociales - aprobada con distinción máxima}
\cventry{2018}{Pontificia Universidad Católica de Chile}{}{}{}{Facultad de Matemática - Diplomado en Data Science}
\cventry{2022}{University of Edinburgh (UK)}{}{}{}{School of Law - Master in Law in Information Technology Law (LL.M) - Tesis: ``\textit{Interoperability in (the right to) data portability: Is it possible to lay down a functional legal criterion?}''}
\subsection{\noun{Otros Cursos de Postítulo}}

\cventry{Verano 2016}{University of Chicago Law School}{}{}{}{Coase Sandor Institute for Law and Economics - \emph{Summer Institute in Law and Economics}}
\cventry{Verano 2021}{Universiteit van Amsterdam}{}{}{}{Institute for Information Law (IViR) \emph{Summer Course on Privacy Law and Policy}}


\section{\noun{Reconocimientos}}

\cventry{2013}{Primer Lugar en concurso de ponencias III Congreso de Derecho Civil}{}{Escuela de Derecho - Universidad de Chile}{Chile}{}
\cventry{2019-2020}{Abogado reconocido en \emph{Data Protection and Privacy }(Chile)}{}{Who's Who Legal}{Reino Unido}{}

\section{\noun{Experiencia Laboral}}


\cventry{2010 - 2013 | 2014 - 2015}{Sateler Depolo Diemoz Abogados (actual \emph{Kennedys Law Chile})}{}{}{}{ Abogado Asociado. Mi trabajo estuvo especialmente enfocado en proveer servicios legales a \emph{retailers}, emisores de tarjetas de crédito no bancarias, compañías de seguros y empresas de tecnología financiera en asuntos relativos a protección al consumidor, regulación financiera y protección de datos personales.}

\cventry{Octubre 2015 - Abril 2017}{Superintendencia de Bancos e Instituciones Financieras (SBIF)}{}{}{}{Abogado en Dirección Jurídica - Unidad de Cooperativas y Entidades No Bancarias. Como parte de mi trabajo fuí designado a la unidad dedicada a la regulación de emisores y operadores de tarjetas de pago (débito, crédito y prepago) cooperativas de ahorro y crédito, SAG relacionadas a sistemas de pago, y otras instituciones financieras no bancarias (NBFIs). Asimismo, tuve una participación activa en la revisión de las normativas de externalización de servicios, incidentes operacionales y continuidad de negocios (RAN 20-7, 20-8 y 20-9) y en el Grupo de Trabajo Interdisciplinario de Medios de Pago, y la Mesa Interministerial de Ciberseguridad}

\cventry{Abril 2017 - Febrero  2020}{FerradaNehme Abogados}{}{}{}{Asociado Senior adscrito a los grupos de Regulación Económica, Derecho del Consumo, y Tecnología, Medios y Telecomunicaciones (TMT). Mi labor se enfocó en proveer asesoría preventiva y transaccional a \emph{retailers}, entidades financieras, emprendimientos Fintech, aseguradores, compañías de telecomunicaciones y plataformas de comercio electrónico -entre otras-, en aspectos tales como regulación financiera, protección de datos personales, derecho del consumo local y transfronterizo, y aplicaciones de transformación digital e inteligencia artificial en procesos productivos y modelos de negocios disruptivos.}

\cventry{Febrero  2020 - \emph{presente}}{Clavis Legal Abogados}{}{}{}{Socio líder del área de regulación financiera, tecnologías de la información, e innovación. Me desempeño como asesor y consultor legal en materias de regulación financiera, comercio electrónico, protección de datos personales, portabilidad financiera, sistemas de cumplimiento normativo, entre otras temáticas afines. Dentro del portafolio de clientes se encuentran \textit{retailers}, empresas fintech, instituciones financieras, plataformas de comercio electrónico, e integradores de aplicativos y brókeres de datos.} 



\section{\noun{Docencia y Relatoría}}


\cventry{2010}{Subsecretaria de Telecomunicaciones}{}{}{}{Relator en curso de capacitación ``Regulación Económica aplicada al Mercado de las Telecomunicaciones''.}

\cventry{Otoño 2014 | Otoño 2015}{Facultad de Derecho Universidad Adolfo Ibañez}{}{}{}{Ayudante en curso ``Derecho del Consumo'' del programa de Magíster en Derecho Privado.}

\cventry{2019-2020}{Facultad de Matemáticas Pontificia Universidad Católica de Chile}{}{}{}{Profesor del curso de ``Protección de Datos Personales para Data Science'' del Diplomado de Data Science}
\cventry{2020}{Comisión para el Mercado Financiero}{Dirección de Conducta de Mercado}{}{}{Relator del curso \textit{``Desarrollos Fintech y Nuevo Ecosistema de Pagos: Desafíos para la Regulación y Supervisión Financiera''}.}
\cventry{2021}{Superintendencia de Pensiones}{}{}{}{Relator del curso \textit{``Protección de Datos Personales: Escenario actual y Nuevos Desafíos''.}}


\renewcommand{\refname}{\textsc{Publicaciones y Ponencias}}
\nocite{*}
\bibliographystyle{uchile1url3}
\bibliography{publications}                        

\section{\noun{Afiliaciones}}

\cvlistitem{IACL (\emph{International Association of Consumer Law})}
\cvlistitem{IAPP (\emph{International Association of Privacy Professionals})}

\section{\noun{Investigación Institucional}}


\cventry{2009 - 2010}{Observatorio Fucatel}{}{}{}{ Co-investigador en proyecto de análisis del marco jurídico para la introducción de la TDT para el Ministerio Secretaria General
de Gobierno (SEGEGOB). Título de informe final: \emph{Gestión de espectro radioeléctrico, mercado secundario y liberalización asignativa.}}


\cventry{Otoño 2010 - Verano 2012}{Departamento de Ingeniería Industrial - Universidad de Chile}{}{}{}{ Investigador en el proyecto titulado \emph{ ``Web mining and privacy concerns''.}}


\cventry{Primavera 2014}{Centro Regulación y Competencia Universidad de Chile (RegCom)}{}{}{}{ Investigador externo en análisis comparado de normativa de protección al consumidor.}

\section{\noun{Habilidades y Certificaciones Técnicas}}

\cvitemwithcomment{Idiomas}{Español (nativo), Inglés (avanzado - TOEFL Score 106)}{}
\cvitemwithcomment{Normas y Estándares Técnicos}{Implementador de estándares ISO 31.000, ISO/IEC 29.100 e ISO/IEC 27.701}{}
\cvitemwithcomment{Informática}{Ms. Office Apps (avanzado), \LaTeX (intermedio), R (intermedio), Python (básico)}{}
\end{document}
